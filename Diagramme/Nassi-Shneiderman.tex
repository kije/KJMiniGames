% !TEX encoding = UTF-8 Unicode
\documentclass[11pt,twoside]{doc}
\usepackage{nassi}

\hyphenation{sta-te-ment}
\begin{document}

\title{KJMiniGames Nassi-Shneiderman Diagramme}
\author{Kim Jeker}
\date{}
\maketitle


\nassiwidth=\columnwidth\setiftext {Ja }{ Nein}





\STRUCT{Main}{in ControllerClass.java}{%
  \ACTION{Konstruktor von ControllerClass aufrufen}%
}%

\STRUCT{prepare()}{in KJMineSweeper.java}{%
	\ACTION{Z{\"a}hlervariable f{\"u}r Minen auf 0 initialisieren}%
	 \WHILE{F{\"u}r jede "Reihe" im "Minen"-Array ...}{
  		\WHILE{F{\"u}r jede "Zelle" in der Reihe ...}{
			\ACTION{... Neues Feld initialisieren}%
			\ACTION{... Positionsvariable des Feldes setzen}%
			\ACTION{... "Pseudeo"-Zuf{\"a}llig entscheiden, ob Feld eine Mine ist und dem entsprechend dessen isMine-Variable setzen }%
			\IF{Feld ist eine Mine}\THEN{
				\ACTION{Erh{\"o}he Z{\"a}hlervariable f{\"u}r Minen um 1}%
			}\ELSE{
				\ACTION{}%
			}\ENDIF%
			\ACTION{Den ActionListener auf die eigene Klasse (this) setzten}%
			\ACTION{Das Feld dem Spielfeld hinzufügen}%
		}\ENDWHILE%
	}\ENDWHILE%
	\ACTION{Die Anzahl Mienen auf der Konsole ausgeben}%
	\WHILE{F{\"u}r jede "Reihe" im "Minen"-Array ...}{
  		\WHILE{F{\"u}r jede "Zelle" in der Reihe ...}{
			\ACTION{... die Zellen rings herum analysieren (methode)}%
		}\ENDWHILE%
	}\ENDWHILE%
}%


\end{document}
